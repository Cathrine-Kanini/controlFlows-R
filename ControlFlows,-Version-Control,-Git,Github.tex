% Options for packages loaded elsewhere
\PassOptionsToPackage{unicode}{hyperref}
\PassOptionsToPackage{hyphens}{url}
\documentclass[
]{article}
\usepackage{xcolor}
\usepackage[margin=1in]{geometry}
\usepackage{amsmath,amssymb}
\setcounter{secnumdepth}{-\maxdimen} % remove section numbering
\usepackage{iftex}
\ifPDFTeX
  \usepackage[T1]{fontenc}
  \usepackage[utf8]{inputenc}
  \usepackage{textcomp} % provide euro and other symbols
\else % if luatex or xetex
  \usepackage{unicode-math} % this also loads fontspec
  \defaultfontfeatures{Scale=MatchLowercase}
  \defaultfontfeatures[\rmfamily]{Ligatures=TeX,Scale=1}
\fi
\usepackage{lmodern}
\ifPDFTeX\else
  % xetex/luatex font selection
\fi
% Use upquote if available, for straight quotes in verbatim environments
\IfFileExists{upquote.sty}{\usepackage{upquote}}{}
\IfFileExists{microtype.sty}{% use microtype if available
  \usepackage[]{microtype}
  \UseMicrotypeSet[protrusion]{basicmath} % disable protrusion for tt fonts
}{}
\makeatletter
\@ifundefined{KOMAClassName}{% if non-KOMA class
  \IfFileExists{parskip.sty}{%
    \usepackage{parskip}
  }{% else
    \setlength{\parindent}{0pt}
    \setlength{\parskip}{6pt plus 2pt minus 1pt}}
}{% if KOMA class
  \KOMAoptions{parskip=half}}
\makeatother
\usepackage{color}
\usepackage{fancyvrb}
\newcommand{\VerbBar}{|}
\newcommand{\VERB}{\Verb[commandchars=\\\{\}]}
\DefineVerbatimEnvironment{Highlighting}{Verbatim}{commandchars=\\\{\}}
% Add ',fontsize=\small' for more characters per line
\usepackage{framed}
\definecolor{shadecolor}{RGB}{248,248,248}
\newenvironment{Shaded}{\begin{snugshade}}{\end{snugshade}}
\newcommand{\AlertTok}[1]{\textcolor[rgb]{0.94,0.16,0.16}{#1}}
\newcommand{\AnnotationTok}[1]{\textcolor[rgb]{0.56,0.35,0.01}{\textbf{\textit{#1}}}}
\newcommand{\AttributeTok}[1]{\textcolor[rgb]{0.13,0.29,0.53}{#1}}
\newcommand{\BaseNTok}[1]{\textcolor[rgb]{0.00,0.00,0.81}{#1}}
\newcommand{\BuiltInTok}[1]{#1}
\newcommand{\CharTok}[1]{\textcolor[rgb]{0.31,0.60,0.02}{#1}}
\newcommand{\CommentTok}[1]{\textcolor[rgb]{0.56,0.35,0.01}{\textit{#1}}}
\newcommand{\CommentVarTok}[1]{\textcolor[rgb]{0.56,0.35,0.01}{\textbf{\textit{#1}}}}
\newcommand{\ConstantTok}[1]{\textcolor[rgb]{0.56,0.35,0.01}{#1}}
\newcommand{\ControlFlowTok}[1]{\textcolor[rgb]{0.13,0.29,0.53}{\textbf{#1}}}
\newcommand{\DataTypeTok}[1]{\textcolor[rgb]{0.13,0.29,0.53}{#1}}
\newcommand{\DecValTok}[1]{\textcolor[rgb]{0.00,0.00,0.81}{#1}}
\newcommand{\DocumentationTok}[1]{\textcolor[rgb]{0.56,0.35,0.01}{\textbf{\textit{#1}}}}
\newcommand{\ErrorTok}[1]{\textcolor[rgb]{0.64,0.00,0.00}{\textbf{#1}}}
\newcommand{\ExtensionTok}[1]{#1}
\newcommand{\FloatTok}[1]{\textcolor[rgb]{0.00,0.00,0.81}{#1}}
\newcommand{\FunctionTok}[1]{\textcolor[rgb]{0.13,0.29,0.53}{\textbf{#1}}}
\newcommand{\ImportTok}[1]{#1}
\newcommand{\InformationTok}[1]{\textcolor[rgb]{0.56,0.35,0.01}{\textbf{\textit{#1}}}}
\newcommand{\KeywordTok}[1]{\textcolor[rgb]{0.13,0.29,0.53}{\textbf{#1}}}
\newcommand{\NormalTok}[1]{#1}
\newcommand{\OperatorTok}[1]{\textcolor[rgb]{0.81,0.36,0.00}{\textbf{#1}}}
\newcommand{\OtherTok}[1]{\textcolor[rgb]{0.56,0.35,0.01}{#1}}
\newcommand{\PreprocessorTok}[1]{\textcolor[rgb]{0.56,0.35,0.01}{\textit{#1}}}
\newcommand{\RegionMarkerTok}[1]{#1}
\newcommand{\SpecialCharTok}[1]{\textcolor[rgb]{0.81,0.36,0.00}{\textbf{#1}}}
\newcommand{\SpecialStringTok}[1]{\textcolor[rgb]{0.31,0.60,0.02}{#1}}
\newcommand{\StringTok}[1]{\textcolor[rgb]{0.31,0.60,0.02}{#1}}
\newcommand{\VariableTok}[1]{\textcolor[rgb]{0.00,0.00,0.00}{#1}}
\newcommand{\VerbatimStringTok}[1]{\textcolor[rgb]{0.31,0.60,0.02}{#1}}
\newcommand{\WarningTok}[1]{\textcolor[rgb]{0.56,0.35,0.01}{\textbf{\textit{#1}}}}
\usepackage{longtable,booktabs,array}
\usepackage{calc} % for calculating minipage widths
% Correct order of tables after \paragraph or \subparagraph
\usepackage{etoolbox}
\makeatletter
\patchcmd\longtable{\par}{\if@noskipsec\mbox{}\fi\par}{}{}
\makeatother
% Allow footnotes in longtable head/foot
\IfFileExists{footnotehyper.sty}{\usepackage{footnotehyper}}{\usepackage{footnote}}
\makesavenoteenv{longtable}
\usepackage{graphicx}
\makeatletter
\newsavebox\pandoc@box
\newcommand*\pandocbounded[1]{% scales image to fit in text height/width
  \sbox\pandoc@box{#1}%
  \Gscale@div\@tempa{\textheight}{\dimexpr\ht\pandoc@box+\dp\pandoc@box\relax}%
  \Gscale@div\@tempb{\linewidth}{\wd\pandoc@box}%
  \ifdim\@tempb\p@<\@tempa\p@\let\@tempa\@tempb\fi% select the smaller of both
  \ifdim\@tempa\p@<\p@\scalebox{\@tempa}{\usebox\pandoc@box}%
  \else\usebox{\pandoc@box}%
  \fi%
}
% Set default figure placement to htbp
\def\fps@figure{htbp}
\makeatother
\setlength{\emergencystretch}{3em} % prevent overfull lines
\providecommand{\tightlist}{%
  \setlength{\itemsep}{0pt}\setlength{\parskip}{0pt}}
\usepackage{bookmark}
\IfFileExists{xurl.sty}{\usepackage{xurl}}{} % add URL line breaks if available
\urlstyle{same}
\hypersetup{
  pdftitle={Control Flow, Version Control, Git \& GitHub with R Programming},
  pdfauthor={Cathrine Kanini},
  hidelinks,
  pdfcreator={LaTeX via pandoc}}

\title{Control Flow, Version Control, Git \& GitHub with R Programming}
\author{Cathrine Kanini}
\date{2026-01-19}

\begin{document}
\maketitle

{
\setcounter{tocdepth}{3}
\tableofcontents
}
\section{Part 1: Control Flow in R
Programming}\label{part-1-control-flow-in-r-programming}

Control flow structures allow us to control the execution of code based
on conditions and repetitions.

\begin{center}\rule{0.5\linewidth}{0.5pt}\end{center}

\subsection{1. Decision Making
Structures}\label{decision-making-structures}

Decision making in programming allows us to control the flow of
execution based on specific conditions. In R, various decision-making
structures help us execute statements conditionally. These include:

\subsubsection{1.1 if Statement}\label{if-statement}

The \texttt{if} statement evaluates a condition and executes code only
if the condition is TRUE.

\textbf{Syntax:}

\begin{Shaded}
\begin{Highlighting}[]
\ControlFlowTok{if}\NormalTok{ (condition) \{}
  \CommentTok{\# execute this code}
\NormalTok{\}}
\end{Highlighting}
\end{Shaded}

\textbf{Example:}

\begin{Shaded}
\begin{Highlighting}[]
\NormalTok{a }\OtherTok{\textless{}{-}} \DecValTok{76}
\NormalTok{b }\OtherTok{\textless{}{-}} \DecValTok{67}

\ControlFlowTok{if}\NormalTok{ (a }\SpecialCharTok{\textgreater{}}\NormalTok{ b) \{}
\NormalTok{  c }\OtherTok{\textless{}{-}}\NormalTok{ a }\SpecialCharTok{{-}}\NormalTok{ b}
  \FunctionTok{print}\NormalTok{(}\StringTok{"condition a \textgreater{} b is TRUE"}\NormalTok{)}
  \FunctionTok{print}\NormalTok{(}\FunctionTok{paste}\NormalTok{(}\StringTok{"Difference between a, b is:"}\NormalTok{, c))}
\NormalTok{\}}
\end{Highlighting}
\end{Shaded}

\begin{verbatim}
#> [1] "condition a > b is TRUE"
#> [1] "Difference between a, b is: 9"
\end{verbatim}

\begin{Shaded}
\begin{Highlighting}[]
\ControlFlowTok{if}\NormalTok{ (a }\SpecialCharTok{\textless{}}\NormalTok{ b) \{}
\NormalTok{  c }\OtherTok{\textless{}{-}}\NormalTok{ a }\SpecialCharTok{{-}}\NormalTok{ b}
  \FunctionTok{print}\NormalTok{(}\StringTok{"condition a \textless{} b is TRUE"}\NormalTok{)}
  \FunctionTok{print}\NormalTok{(}\FunctionTok{paste}\NormalTok{(}\StringTok{"Difference between a, b is:"}\NormalTok{, c))}
\NormalTok{\}}
\end{Highlighting}
\end{Shaded}

\textbf{Key Points:}

\begin{itemize}
\tightlist
\item
  Only executes when condition is TRUE
\item
  No alternative action if FALSE
\item
  Can have multiple independent if statements
\end{itemize}

\begin{center}\rule{0.5\linewidth}{0.5pt}\end{center}

\subsubsection{1.2 if-else Statement}\label{if-else-statement}

The \texttt{if-else} statement provides an alternative action when the
condition is FALSE.

\textbf{Syntax:}

\begin{Shaded}
\begin{Highlighting}[]
\ControlFlowTok{if}\NormalTok{ (condition) \{}
  \CommentTok{\# execute if TRUE}
\NormalTok{\} }\ControlFlowTok{else}\NormalTok{ \{}
  \CommentTok{\# execute if FALSE}
\NormalTok{\}}
\end{Highlighting}
\end{Shaded}

\textbf{Example:}

\begin{Shaded}
\begin{Highlighting}[]
\NormalTok{a }\OtherTok{\textless{}{-}} \DecValTok{67}
\NormalTok{b }\OtherTok{\textless{}{-}} \DecValTok{76}

\ControlFlowTok{if}\NormalTok{ (a }\SpecialCharTok{\textgreater{}}\NormalTok{ b) \{}
\NormalTok{  c }\OtherTok{\textless{}{-}}\NormalTok{ a }\SpecialCharTok{{-}}\NormalTok{ b}
  \FunctionTok{print}\NormalTok{(}\StringTok{"condition a \textgreater{} b is TRUE"}\NormalTok{)}
  \FunctionTok{print}\NormalTok{(}\FunctionTok{paste}\NormalTok{(}\StringTok{"Difference between a, b is:"}\NormalTok{, c))}
\NormalTok{\} }\ControlFlowTok{else}\NormalTok{ \{}
\NormalTok{  c }\OtherTok{\textless{}{-}}\NormalTok{ a }\SpecialCharTok{{-}}\NormalTok{ b}
  \FunctionTok{print}\NormalTok{(}\StringTok{"condition a \textgreater{} b is FALSE"}\NormalTok{)}
  \FunctionTok{print}\NormalTok{(}\FunctionTok{paste}\NormalTok{(}\StringTok{"Difference between a, b is:"}\NormalTok{, c))}
\NormalTok{\}}
\end{Highlighting}
\end{Shaded}

\begin{verbatim}
#> [1] "condition a > b is FALSE"
#> [1] "Difference between a, b is: -9"
\end{verbatim}

\textbf{Practical Example: Grade Classification}

\begin{Shaded}
\begin{Highlighting}[]
\NormalTok{score }\OtherTok{\textless{}{-}} \DecValTok{85}

\ControlFlowTok{if}\NormalTok{ (score }\SpecialCharTok{\textgreater{}=} \DecValTok{90}\NormalTok{) \{}
\NormalTok{  grade }\OtherTok{\textless{}{-}} \StringTok{"A"}
\NormalTok{\} }\ControlFlowTok{else}\NormalTok{ \{}
\NormalTok{  grade }\OtherTok{\textless{}{-}} \StringTok{"B or lower"}
\NormalTok{\}}

\FunctionTok{print}\NormalTok{(}\FunctionTok{paste}\NormalTok{(}\StringTok{"Your grade is:"}\NormalTok{, grade))}
\end{Highlighting}
\end{Shaded}

\begin{verbatim}
#> [1] "Your grade is: B or lower"
\end{verbatim}

\begin{center}\rule{0.5\linewidth}{0.5pt}\end{center}

\subsubsection{1.3 if-else-if Ladder}\label{if-else-if-ladder}

Multiple conditions are evaluated sequentially until one is TRUE.

\textbf{Syntax:}

\begin{Shaded}
\begin{Highlighting}[]
\ControlFlowTok{if}\NormalTok{ (condition1) \{}
  \CommentTok{\# execute if condition1 is TRUE}
\NormalTok{\} }\ControlFlowTok{else} \ControlFlowTok{if}\NormalTok{ (condition2) \{}
  \CommentTok{\# execute if condition2 is TRUE}
\NormalTok{\} }\ControlFlowTok{else}\NormalTok{ \{}
  \CommentTok{\# execute if all conditions are FALSE}
\NormalTok{\}}
\end{Highlighting}
\end{Shaded}

\textbf{Example:}

\begin{Shaded}
\begin{Highlighting}[]
\NormalTok{a }\OtherTok{\textless{}{-}} \DecValTok{67}
\NormalTok{b }\OtherTok{\textless{}{-}} \DecValTok{76}
\NormalTok{c }\OtherTok{\textless{}{-}} \DecValTok{99}

\ControlFlowTok{if}\NormalTok{ (a }\SpecialCharTok{\textgreater{}}\NormalTok{ b }\SpecialCharTok{\&\&}\NormalTok{ b }\SpecialCharTok{\textgreater{}}\NormalTok{ c) \{}
  \FunctionTok{print}\NormalTok{(}\StringTok{"condition a \textgreater{} b \textgreater{} c is TRUE"}\NormalTok{)}
\NormalTok{\} }\ControlFlowTok{else} \ControlFlowTok{if}\NormalTok{ (a }\SpecialCharTok{\textless{}}\NormalTok{ b }\SpecialCharTok{\&\&}\NormalTok{ b }\SpecialCharTok{\textgreater{}}\NormalTok{ c) \{}
  \FunctionTok{print}\NormalTok{(}\StringTok{"condition a \textless{} b \textgreater{} c is TRUE"}\NormalTok{)}
\NormalTok{\} }\ControlFlowTok{else} \ControlFlowTok{if}\NormalTok{ (a }\SpecialCharTok{\textless{}}\NormalTok{ b }\SpecialCharTok{\&\&}\NormalTok{ b }\SpecialCharTok{\textless{}}\NormalTok{ c) \{}
  \FunctionTok{print}\NormalTok{(}\StringTok{"condition a \textless{} b \textless{} c is TRUE"}\NormalTok{)}
\NormalTok{\}}
\end{Highlighting}
\end{Shaded}

\begin{verbatim}
#> [1] "condition a < b < c is TRUE"
\end{verbatim}

\textbf{Practical Example: Complete Grade System}

\begin{Shaded}
\begin{Highlighting}[]
\NormalTok{score }\OtherTok{\textless{}{-}} \DecValTok{78}

\ControlFlowTok{if}\NormalTok{ (score }\SpecialCharTok{\textgreater{}=} \DecValTok{90}\NormalTok{) \{}
\NormalTok{  grade }\OtherTok{\textless{}{-}} \StringTok{"A"}
\NormalTok{  message }\OtherTok{\textless{}{-}} \StringTok{"Excellent!"}
\NormalTok{\} }\ControlFlowTok{else} \ControlFlowTok{if}\NormalTok{ (score }\SpecialCharTok{\textgreater{}=} \DecValTok{80}\NormalTok{) \{}
\NormalTok{  grade }\OtherTok{\textless{}{-}} \StringTok{"B"}
\NormalTok{  message }\OtherTok{\textless{}{-}} \StringTok{"Very Good!"}
\NormalTok{\} }\ControlFlowTok{else} \ControlFlowTok{if}\NormalTok{ (score }\SpecialCharTok{\textgreater{}=} \DecValTok{70}\NormalTok{) \{}
\NormalTok{  grade }\OtherTok{\textless{}{-}} \StringTok{"C"}
\NormalTok{  message }\OtherTok{\textless{}{-}} \StringTok{"Good!"}
\NormalTok{\} }\ControlFlowTok{else} \ControlFlowTok{if}\NormalTok{ (score }\SpecialCharTok{\textgreater{}=} \DecValTok{60}\NormalTok{) \{}
\NormalTok{  grade }\OtherTok{\textless{}{-}} \StringTok{"D"}
\NormalTok{  message }\OtherTok{\textless{}{-}} \StringTok{"Pass"}
\NormalTok{\} }\ControlFlowTok{else}\NormalTok{ \{}
\NormalTok{  grade }\OtherTok{\textless{}{-}} \StringTok{"F"}
\NormalTok{  message }\OtherTok{\textless{}{-}} \StringTok{"Fail"}
\NormalTok{\}}

\FunctionTok{cat}\NormalTok{(}\FunctionTok{sprintf}\NormalTok{(}\StringTok{"Score: \%d}\SpecialCharTok{\textbackslash{}n}\StringTok{Grade: \%s}\SpecialCharTok{\textbackslash{}n}\StringTok{\%s}\SpecialCharTok{\textbackslash{}n}\StringTok{"}\NormalTok{, score, grade, message))}
\end{Highlighting}
\end{Shaded}

\begin{verbatim}
#> Score: 78
#> Grade: C
#> Good!
\end{verbatim}

\begin{center}\rule{0.5\linewidth}{0.5pt}\end{center}

\subsubsection{1.4 Nested if-else
Statement}\label{nested-if-else-statement}

An if-else statement inside another if-else statement.

\textbf{Syntax:}

\begin{Shaded}
\begin{Highlighting}[]
\ControlFlowTok{if}\NormalTok{ (parent\_condition) \{}
  \ControlFlowTok{if}\NormalTok{ (child\_condition1) \{}
    \CommentTok{\# code}
\NormalTok{  \} }\ControlFlowTok{else}\NormalTok{ \{}
    \CommentTok{\# code}
\NormalTok{  \}}
\NormalTok{\} }\ControlFlowTok{else}\NormalTok{ \{}
  \ControlFlowTok{if}\NormalTok{ (child\_condition2) \{}
    \CommentTok{\# code}
\NormalTok{  \} }\ControlFlowTok{else}\NormalTok{ \{}
    \CommentTok{\# code}
\NormalTok{  \}}
\NormalTok{\}}
\end{Highlighting}
\end{Shaded}

\textbf{Example:}

\begin{Shaded}
\begin{Highlighting}[]
\NormalTok{a }\OtherTok{\textless{}{-}} \DecValTok{10}
\NormalTok{b }\OtherTok{\textless{}{-}} \DecValTok{11}

\ControlFlowTok{if}\NormalTok{ (a }\SpecialCharTok{==} \DecValTok{10}\NormalTok{) \{}
  \ControlFlowTok{if}\NormalTok{ (b }\SpecialCharTok{==} \DecValTok{10}\NormalTok{) \{}
    \FunctionTok{print}\NormalTok{(}\StringTok{"a:10 b:10"}\NormalTok{)}
\NormalTok{  \} }\ControlFlowTok{else}\NormalTok{ \{}
    \FunctionTok{print}\NormalTok{(}\StringTok{"a:10 b:11"}\NormalTok{)}
\NormalTok{  \}}
\NormalTok{\} }\ControlFlowTok{else}\NormalTok{ \{}
  \ControlFlowTok{if}\NormalTok{ (a }\SpecialCharTok{==} \DecValTok{11}\NormalTok{) \{}
    \FunctionTok{print}\NormalTok{(}\StringTok{"a:11 b:10"}\NormalTok{)}
\NormalTok{  \} }\ControlFlowTok{else}\NormalTok{ \{}
    \FunctionTok{print}\NormalTok{(}\StringTok{"a:11 b:11"}\NormalTok{)}
\NormalTok{  \}}
\NormalTok{\}}
\end{Highlighting}
\end{Shaded}

\begin{verbatim}
#> [1] "a:10 b:11"
\end{verbatim}

\textbf{Practical Example: Login System}

\begin{Shaded}
\begin{Highlighting}[]
\NormalTok{username }\OtherTok{\textless{}{-}} \StringTok{"admin"}
\NormalTok{password }\OtherTok{\textless{}{-}} \StringTok{"pass123"}

\ControlFlowTok{if}\NormalTok{ (username }\SpecialCharTok{==} \StringTok{"admin"}\NormalTok{) \{}
  \ControlFlowTok{if}\NormalTok{ (password }\SpecialCharTok{==} \StringTok{"pass123"}\NormalTok{) \{}
    \FunctionTok{print}\NormalTok{(}\StringTok{"✓ Login successful!"}\NormalTok{)}
\NormalTok{  \} }\ControlFlowTok{else}\NormalTok{ \{}
    \FunctionTok{print}\NormalTok{(}\StringTok{"✗ Incorrect password"}\NormalTok{)}
\NormalTok{  \}}
\NormalTok{\} }\ControlFlowTok{else}\NormalTok{ \{}
  \ControlFlowTok{if}\NormalTok{ (username }\SpecialCharTok{==} \StringTok{""}\NormalTok{) \{}
    \FunctionTok{print}\NormalTok{(}\StringTok{"✗ Username cannot be empty"}\NormalTok{)}
\NormalTok{  \} }\ControlFlowTok{else}\NormalTok{ \{}
    \FunctionTok{print}\NormalTok{(}\StringTok{"✗ User not found"}\NormalTok{)}
\NormalTok{  \}}
\NormalTok{\}}
\end{Highlighting}
\end{Shaded}

\begin{verbatim}
#> [1] "✓ Login successful!"
\end{verbatim}

\begin{center}\rule{0.5\linewidth}{0.5pt}\end{center}

\subsubsection{1.5 switch Statement}\label{switch-statement}

Compares an expression against multiple cases and executes matching
code.

\textbf{Syntax:}

\begin{Shaded}
\begin{Highlighting}[]
\ControlFlowTok{switch}\NormalTok{(expression, }
       \AttributeTok{case1 =}\NormalTok{ result1,}
       \AttributeTok{case2 =}\NormalTok{ result2,}
\NormalTok{       default\_result)}
\end{Highlighting}
\end{Shaded}

\textbf{Example with Numbers:}

\begin{Shaded}
\begin{Highlighting}[]
\NormalTok{day\_num }\OtherTok{\textless{}{-}} \DecValTok{3}

\NormalTok{day\_name }\OtherTok{\textless{}{-}} \ControlFlowTok{switch}\NormalTok{(day\_num,}
                   \StringTok{"1"} \OtherTok{=} \StringTok{"Monday"}\NormalTok{,}
                   \StringTok{"2"} \OtherTok{=} \StringTok{"Tuesday"}\NormalTok{,}
                   \StringTok{"3"} \OtherTok{=} \StringTok{"Wednesday"}\NormalTok{,}
                   \StringTok{"4"} \OtherTok{=} \StringTok{"Thursday"}\NormalTok{,}
                   \StringTok{"5"} \OtherTok{=} \StringTok{"Friday"}\NormalTok{,}
                   \StringTok{"6"} \OtherTok{=} \StringTok{"Saturday"}\NormalTok{,}
                   \StringTok{"7"} \OtherTok{=} \StringTok{"Sunday"}\NormalTok{,}
                   \StringTok{"Invalid day"}\NormalTok{)}

\FunctionTok{print}\NormalTok{(}\FunctionTok{paste}\NormalTok{(}\StringTok{"Day"}\NormalTok{, day\_num, }\StringTok{"is"}\NormalTok{, day\_name))}
\end{Highlighting}
\end{Shaded}

\begin{verbatim}
#> [1] "Day 3 is Wednesday"
\end{verbatim}

\textbf{Example with Characters:}

\begin{Shaded}
\begin{Highlighting}[]
\NormalTok{operation }\OtherTok{\textless{}{-}} \StringTok{"add"}

\NormalTok{result }\OtherTok{\textless{}{-}} \ControlFlowTok{switch}\NormalTok{(operation,}
                 \StringTok{"add"} \OtherTok{=} \DecValTok{5} \SpecialCharTok{+} \DecValTok{3}\NormalTok{,}
                 \StringTok{"subtract"} \OtherTok{=} \DecValTok{5} \SpecialCharTok{{-}} \DecValTok{3}\NormalTok{,}
                 \StringTok{"multiply"} \OtherTok{=} \DecValTok{5} \SpecialCharTok{*} \DecValTok{3}\NormalTok{,}
                 \StringTok{"divide"} \OtherTok{=} \DecValTok{5} \SpecialCharTok{/} \DecValTok{3}\NormalTok{,}
                 \StringTok{"Unknown operation"}\NormalTok{)}

\FunctionTok{print}\NormalTok{(}\FunctionTok{paste}\NormalTok{(}\StringTok{"Result:"}\NormalTok{, result))}
\end{Highlighting}
\end{Shaded}

\begin{verbatim}
#> [1] "Result: 8"
\end{verbatim}

\begin{center}\rule{0.5\linewidth}{0.5pt}\end{center}

\subsection{2. Loops in R}\label{loops-in-r}

Loops allow repetitive execution of code blocks.

\subsubsection{2.1 for Loop}\label{for-loop}

Used when the number of iterations is known beforehand.

\textbf{Syntax:}

\begin{Shaded}
\begin{Highlighting}[]
\ControlFlowTok{for}\NormalTok{ (value }\ControlFlowTok{in}\NormalTok{ sequence) \{}
  \CommentTok{\# code to execute}
\NormalTok{\}}
\end{Highlighting}
\end{Shaded}

\textbf{Example 1: Simple Sequence}

\begin{Shaded}
\begin{Highlighting}[]
\ControlFlowTok{for}\NormalTok{ (val }\ControlFlowTok{in} \DecValTok{1}\SpecialCharTok{:}\DecValTok{5}\NormalTok{) \{}
  \FunctionTok{print}\NormalTok{(val)}
\NormalTok{\}}
\end{Highlighting}
\end{Shaded}

\begin{verbatim}
#> [1] 1
#> [1] 2
#> [1] 3
#> [1] 4
#> [1] 5
\end{verbatim}

\textbf{Example 2: Days of the Week}

\begin{Shaded}
\begin{Highlighting}[]
\NormalTok{week }\OtherTok{\textless{}{-}} \FunctionTok{c}\NormalTok{(}\StringTok{\textquotesingle{}Sunday\textquotesingle{}}\NormalTok{, }\StringTok{\textquotesingle{}Monday\textquotesingle{}}\NormalTok{, }\StringTok{\textquotesingle{}Tuesday\textquotesingle{}}\NormalTok{, }\StringTok{\textquotesingle{}Wednesday\textquotesingle{}}\NormalTok{, }
          \StringTok{\textquotesingle{}Thursday\textquotesingle{}}\NormalTok{, }\StringTok{\textquotesingle{}Friday\textquotesingle{}}\NormalTok{, }\StringTok{\textquotesingle{}Saturday\textquotesingle{}}\NormalTok{)}

\ControlFlowTok{for}\NormalTok{ (day }\ControlFlowTok{in}\NormalTok{ week) \{}
  \FunctionTok{print}\NormalTok{(day)}
\NormalTok{\}}
\end{Highlighting}
\end{Shaded}

\begin{verbatim}
#> [1] "Sunday"
#> [1] "Monday"
#> [1] "Tuesday"
#> [1] "Wednesday"
#> [1] "Thursday"
#> [1] "Friday"
#> [1] "Saturday"
\end{verbatim}

\textbf{Example 3: Loop on a List}

\begin{Shaded}
\begin{Highlighting}[]
\NormalTok{my\_list }\OtherTok{\textless{}{-}} \FunctionTok{list}\NormalTok{(}\DecValTok{1}\NormalTok{, }\DecValTok{2}\NormalTok{, }\DecValTok{3}\NormalTok{, }\DecValTok{4}\NormalTok{, }\DecValTok{5}\NormalTok{)}

\ControlFlowTok{for}\NormalTok{ (i }\ControlFlowTok{in} \FunctionTok{seq\_along}\NormalTok{(my\_list)) \{}
\NormalTok{  current\_element }\OtherTok{\textless{}{-}}\NormalTok{ my\_list[[i]]}
  \FunctionTok{print}\NormalTok{(}\FunctionTok{paste}\NormalTok{(}\StringTok{"Element"}\NormalTok{, i, }\StringTok{"is:"}\NormalTok{, current\_element))}
\NormalTok{\}}
\end{Highlighting}
\end{Shaded}

\begin{verbatim}
#> [1] "Element 1 is: 1"
#> [1] "Element 2 is: 2"
#> [1] "Element 3 is: 3"
#> [1] "Element 4 is: 4"
#> [1] "Element 5 is: 5"
\end{verbatim}

\textbf{Example 4: Loop on a Matrix}

\begin{Shaded}
\begin{Highlighting}[]
\NormalTok{my\_matrix }\OtherTok{\textless{}{-}} \FunctionTok{matrix}\NormalTok{(}\DecValTok{1}\SpecialCharTok{:}\DecValTok{9}\NormalTok{, }\AttributeTok{nrow =} \DecValTok{3}\NormalTok{)}
\FunctionTok{print}\NormalTok{(}\StringTok{"Matrix:"}\NormalTok{)}
\end{Highlighting}
\end{Shaded}

\begin{verbatim}
#> [1] "Matrix:"
\end{verbatim}

\begin{Shaded}
\begin{Highlighting}[]
\FunctionTok{print}\NormalTok{(my\_matrix)}
\end{Highlighting}
\end{Shaded}

\begin{verbatim}
#>      [,1] [,2] [,3]
#> [1,]    1    4    7
#> [2,]    2    5    8
#> [3,]    3    6    9
\end{verbatim}

\begin{Shaded}
\begin{Highlighting}[]
\FunctionTok{cat}\NormalTok{(}\StringTok{"}\SpecialCharTok{\textbackslash{}n}\StringTok{Iterating through matrix:}\SpecialCharTok{\textbackslash{}n}\StringTok{"}\NormalTok{)}
\end{Highlighting}
\end{Shaded}

\begin{verbatim}
#> 
#> Iterating through matrix:
\end{verbatim}

\begin{Shaded}
\begin{Highlighting}[]
\ControlFlowTok{for}\NormalTok{ (i }\ControlFlowTok{in} \FunctionTok{seq\_len}\NormalTok{(}\FunctionTok{nrow}\NormalTok{(my\_matrix))) \{}
  \ControlFlowTok{for}\NormalTok{ (j }\ControlFlowTok{in} \FunctionTok{seq\_len}\NormalTok{(}\FunctionTok{ncol}\NormalTok{(my\_matrix))) \{}
\NormalTok{    current\_element }\OtherTok{\textless{}{-}}\NormalTok{ my\_matrix[i, j]}
    \FunctionTok{cat}\NormalTok{(}\FunctionTok{sprintf}\NormalTok{(}\StringTok{"Position [\%d,\%d] = \%d}\SpecialCharTok{\textbackslash{}n}\StringTok{"}\NormalTok{, i, j, current\_element))}
\NormalTok{  \}}
\NormalTok{\}}
\end{Highlighting}
\end{Shaded}

\begin{verbatim}
#> Position [1,1] = 1
#> Position [1,2] = 4
#> Position [1,3] = 7
#> Position [2,1] = 2
#> Position [2,2] = 5
#> Position [2,3] = 8
#> Position [3,1] = 3
#> Position [3,2] = 6
#> Position [3,3] = 9
\end{verbatim}

\textbf{Example 5: Loop on a Data Frame}

\begin{Shaded}
\begin{Highlighting}[]
\NormalTok{my\_dataframe }\OtherTok{\textless{}{-}} \FunctionTok{data.frame}\NormalTok{(}
  \AttributeTok{Name =} \FunctionTok{c}\NormalTok{(}\StringTok{"Joy"}\NormalTok{, }\StringTok{"Juliya"}\NormalTok{, }\StringTok{"Boby"}\NormalTok{, }\StringTok{"Marry"}\NormalTok{),}
  \AttributeTok{Age =} \FunctionTok{c}\NormalTok{(}\DecValTok{40}\NormalTok{, }\DecValTok{25}\NormalTok{, }\DecValTok{19}\NormalTok{, }\DecValTok{55}\NormalTok{),}
  \AttributeTok{Gender =} \FunctionTok{c}\NormalTok{(}\StringTok{"M"}\NormalTok{, }\StringTok{"F"}\NormalTok{, }\StringTok{"M"}\NormalTok{, }\StringTok{"F"}\NormalTok{)}
\NormalTok{)}

\FunctionTok{print}\NormalTok{(}\StringTok{"Data Frame:"}\NormalTok{)}
\end{Highlighting}
\end{Shaded}

\begin{verbatim}
#> [1] "Data Frame:"
\end{verbatim}

\begin{Shaded}
\begin{Highlighting}[]
\FunctionTok{print}\NormalTok{(my\_dataframe)}
\end{Highlighting}
\end{Shaded}

\begin{verbatim}
#>     Name Age Gender
#> 1    Joy  40      M
#> 2 Juliya  25      F
#> 3   Boby  19      M
#> 4  Marry  55      F
\end{verbatim}

\begin{Shaded}
\begin{Highlighting}[]
\FunctionTok{cat}\NormalTok{(}\StringTok{"}\SpecialCharTok{\textbackslash{}n}\StringTok{Iterating through rows:}\SpecialCharTok{\textbackslash{}n}\StringTok{"}\NormalTok{)}
\end{Highlighting}
\end{Shaded}

\begin{verbatim}
#> 
#> Iterating through rows:
\end{verbatim}

\begin{Shaded}
\begin{Highlighting}[]
\ControlFlowTok{for}\NormalTok{ (i }\ControlFlowTok{in} \FunctionTok{seq\_len}\NormalTok{(}\FunctionTok{nrow}\NormalTok{(my\_dataframe))) \{}
\NormalTok{  current\_row }\OtherTok{\textless{}{-}}\NormalTok{ my\_dataframe[i, ]}
  \FunctionTok{cat}\NormalTok{(}\FunctionTok{sprintf}\NormalTok{(}\StringTok{"Row \%d: \%s, Age \%d, Gender \%s}\SpecialCharTok{\textbackslash{}n}\StringTok{"}\NormalTok{, }
\NormalTok{              i, current\_row}\SpecialCharTok{$}\NormalTok{Name, current\_row}\SpecialCharTok{$}\NormalTok{Age, current\_row}\SpecialCharTok{$}\NormalTok{Gender))}
\NormalTok{\}}
\end{Highlighting}
\end{Shaded}

\begin{verbatim}
#> Row 1: Joy, Age 40, Gender M
#> Row 2: Juliya, Age 25, Gender F
#> Row 3: Boby, Age 19, Gender M
#> Row 4: Marry, Age 55, Gender F
\end{verbatim}

\begin{center}\rule{0.5\linewidth}{0.5pt}\end{center}

\subsubsection{2.2 while Loop}\label{while-loop}

Runs as long as a condition is TRUE. Number of iterations unknown
beforehand.

\textbf{Syntax:}

\begin{Shaded}
\begin{Highlighting}[]
\ControlFlowTok{while}\NormalTok{ (condition) \{}
  \CommentTok{\# code to execute}
\NormalTok{\}}
\end{Highlighting}
\end{Shaded}

\textbf{Example 1: Display Numbers 1 to 5}

\begin{Shaded}
\begin{Highlighting}[]
\NormalTok{val }\OtherTok{\textless{}{-}} \DecValTok{1}
\ControlFlowTok{while}\NormalTok{ (val }\SpecialCharTok{\textless{}=} \DecValTok{5}\NormalTok{) \{}
  \FunctionTok{print}\NormalTok{(val)}
\NormalTok{  val }\OtherTok{\textless{}{-}}\NormalTok{ val }\SpecialCharTok{+} \DecValTok{1}
\NormalTok{\}}
\end{Highlighting}
\end{Shaded}

\begin{verbatim}
#> [1] 1
#> [1] 2
#> [1] 3
#> [1] 4
#> [1] 5
\end{verbatim}

\textbf{Example 2: Calculate Factorial}

\begin{Shaded}
\begin{Highlighting}[]
\NormalTok{n }\OtherTok{\textless{}{-}} \DecValTok{5}
\NormalTok{factorial }\OtherTok{\textless{}{-}} \DecValTok{1}
\NormalTok{i }\OtherTok{\textless{}{-}} \DecValTok{1}

\ControlFlowTok{while}\NormalTok{ (i }\SpecialCharTok{\textless{}=}\NormalTok{ n) \{}
\NormalTok{  factorial }\OtherTok{\textless{}{-}}\NormalTok{ factorial }\SpecialCharTok{*}\NormalTok{ i}
\NormalTok{  i }\OtherTok{\textless{}{-}}\NormalTok{ i }\SpecialCharTok{+} \DecValTok{1}
\NormalTok{\}}

\FunctionTok{print}\NormalTok{(}\FunctionTok{paste}\NormalTok{(}\StringTok{"Factorial of"}\NormalTok{, n, }\StringTok{"is"}\NormalTok{, factorial))}
\end{Highlighting}
\end{Shaded}

\begin{verbatim}
#> [1] "Factorial of 5 is 120"
\end{verbatim}

\textbf{Example 3: Input Validation (Simulation)}

\begin{Shaded}
\begin{Highlighting}[]
\NormalTok{attempts }\OtherTok{\textless{}{-}} \DecValTok{0}
\NormalTok{max\_attempts }\OtherTok{\textless{}{-}} \DecValTok{3}
\NormalTok{success }\OtherTok{\textless{}{-}} \ConstantTok{FALSE}

\ControlFlowTok{while}\NormalTok{ (attempts }\SpecialCharTok{\textless{}}\NormalTok{ max\_attempts }\SpecialCharTok{\&\&} \SpecialCharTok{!}\NormalTok{success) \{}
\NormalTok{  attempts }\OtherTok{\textless{}{-}}\NormalTok{ attempts }\SpecialCharTok{+} \DecValTok{1}
  \CommentTok{\# Simulate user input (would normally use readline())}
\NormalTok{  input\_value }\OtherTok{\textless{}{-}} \FunctionTok{sample}\NormalTok{(}\FunctionTok{c}\NormalTok{(}\ConstantTok{TRUE}\NormalTok{, }\ConstantTok{FALSE}\NormalTok{), }\DecValTok{1}\NormalTok{)}
  
  \ControlFlowTok{if}\NormalTok{ (input\_value) \{}
\NormalTok{    success }\OtherTok{\textless{}{-}} \ConstantTok{TRUE}
    \FunctionTok{print}\NormalTok{(}\FunctionTok{paste}\NormalTok{(}\StringTok{"✓ Success on attempt"}\NormalTok{, attempts))}
\NormalTok{  \} }\ControlFlowTok{else}\NormalTok{ \{}
    \FunctionTok{print}\NormalTok{(}\FunctionTok{paste}\NormalTok{(}\StringTok{"✗ Failed attempt"}\NormalTok{, attempts))}
\NormalTok{  \}}
\NormalTok{\}}
\end{Highlighting}
\end{Shaded}

\begin{verbatim}
#> [1] "✓ Success on attempt 1"
\end{verbatim}

\begin{Shaded}
\begin{Highlighting}[]
\ControlFlowTok{if}\NormalTok{ (}\SpecialCharTok{!}\NormalTok{success) \{}
  \FunctionTok{print}\NormalTok{(}\StringTok{"Maximum attempts reached!"}\NormalTok{)}
\NormalTok{\}}
\end{Highlighting}
\end{Shaded}

\begin{center}\rule{0.5\linewidth}{0.5pt}\end{center}

\subsubsection{2.3 repeat Loop}\label{repeat-loop}

Executes indefinitely until explicitly stopped with \texttt{break}.

\textbf{Syntax:}

\begin{Shaded}
\begin{Highlighting}[]
\ControlFlowTok{repeat}\NormalTok{ \{}
  \CommentTok{\# code to execute}
  
  \ControlFlowTok{if}\NormalTok{ (condition) \{}
    \ControlFlowTok{break}
\NormalTok{  \}}
\NormalTok{\}}
\end{Highlighting}
\end{Shaded}

\textbf{Example 1: Display Numbers 1 to 5}

\begin{Shaded}
\begin{Highlighting}[]
\NormalTok{val }\OtherTok{\textless{}{-}} \DecValTok{1}

\ControlFlowTok{repeat}\NormalTok{ \{}
  \FunctionTok{print}\NormalTok{(val)}
\NormalTok{  val }\OtherTok{\textless{}{-}}\NormalTok{ val }\SpecialCharTok{+} \DecValTok{1}
  
  \ControlFlowTok{if}\NormalTok{ (val }\SpecialCharTok{\textgreater{}} \DecValTok{5}\NormalTok{) \{}
    \ControlFlowTok{break}
\NormalTok{  \}}
\NormalTok{\}}
\end{Highlighting}
\end{Shaded}

\begin{verbatim}
#> [1] 1
#> [1] 2
#> [1] 3
#> [1] 4
#> [1] 5
\end{verbatim}

\textbf{Example 2: Repeat Statement}

\begin{Shaded}
\begin{Highlighting}[]
\NormalTok{i }\OtherTok{\textless{}{-}} \DecValTok{0}

\ControlFlowTok{repeat}\NormalTok{ \{}
  \FunctionTok{print}\NormalTok{(}\StringTok{"Geeks 4 geeks!"}\NormalTok{)}
\NormalTok{  i }\OtherTok{\textless{}{-}}\NormalTok{ i }\SpecialCharTok{+} \DecValTok{1}
  
  \ControlFlowTok{if}\NormalTok{ (i }\SpecialCharTok{==} \DecValTok{5}\NormalTok{) \{}
    \ControlFlowTok{break}
\NormalTok{  \}}
\NormalTok{\}}
\end{Highlighting}
\end{Shaded}

\begin{verbatim}
#> [1] "Geeks 4 geeks!"
#> [1] "Geeks 4 geeks!"
#> [1] "Geeks 4 geeks!"
#> [1] "Geeks 4 geeks!"
#> [1] "Geeks 4 geeks!"
\end{verbatim}

\textbf{Example 3: Menu System}

\begin{Shaded}
\begin{Highlighting}[]
\NormalTok{counter }\OtherTok{\textless{}{-}} \DecValTok{0}

\ControlFlowTok{repeat}\NormalTok{ \{}
\NormalTok{  counter }\OtherTok{\textless{}{-}}\NormalTok{ counter }\SpecialCharTok{+} \DecValTok{1}
  
  \CommentTok{\# Simulate menu selection}
\NormalTok{  choice }\OtherTok{\textless{}{-}} \FunctionTok{sample}\NormalTok{(}\DecValTok{1}\SpecialCharTok{:}\DecValTok{4}\NormalTok{, }\DecValTok{1}\NormalTok{)}
  
  \FunctionTok{cat}\NormalTok{(}\FunctionTok{sprintf}\NormalTok{(}\StringTok{"}\SpecialCharTok{\textbackslash{}n}\StringTok{Iteration \%d {-} Choice: \%d}\SpecialCharTok{\textbackslash{}n}\StringTok{"}\NormalTok{, counter, choice))}
  
  \ControlFlowTok{if}\NormalTok{ (choice }\SpecialCharTok{==} \DecValTok{1}\NormalTok{) \{}
    \FunctionTok{print}\NormalTok{(}\StringTok{"Option 1 selected"}\NormalTok{)}
\NormalTok{  \} }\ControlFlowTok{else} \ControlFlowTok{if}\NormalTok{ (choice }\SpecialCharTok{==} \DecValTok{2}\NormalTok{) \{}
    \FunctionTok{print}\NormalTok{(}\StringTok{"Option 2 selected"}\NormalTok{)}
\NormalTok{  \} }\ControlFlowTok{else} \ControlFlowTok{if}\NormalTok{ (choice }\SpecialCharTok{==} \DecValTok{3}\NormalTok{) \{}
    \FunctionTok{print}\NormalTok{(}\StringTok{"Option 3 selected {-} Exiting"}\NormalTok{)}
    \ControlFlowTok{break}
\NormalTok{  \} }\ControlFlowTok{else}\NormalTok{ \{}
    \FunctionTok{print}\NormalTok{(}\StringTok{"Invalid option"}\NormalTok{)}
\NormalTok{  \}}
  
  \CommentTok{\# Safety break after 10 iterations}
  \ControlFlowTok{if}\NormalTok{ (counter }\SpecialCharTok{\textgreater{}=} \DecValTok{10}\NormalTok{) \{}
    \FunctionTok{print}\NormalTok{(}\StringTok{"Max iterations reached"}\NormalTok{)}
    \ControlFlowTok{break}
\NormalTok{  \}}
\NormalTok{\}}
\end{Highlighting}
\end{Shaded}

\begin{verbatim}
#> 
#> Iteration 1 - Choice: 4
#> [1] "Invalid option"
#> 
#> Iteration 2 - Choice: 2
#> [1] "Option 2 selected"
#> 
#> Iteration 3 - Choice: 2
#> [1] "Option 2 selected"
#> 
#> Iteration 4 - Choice: 3
#> [1] "Option 3 selected - Exiting"
\end{verbatim}

\begin{center}\rule{0.5\linewidth}{0.5pt}\end{center}

\subsection{3. Loop Control Statements}\label{loop-control-statements}

\subsubsection{3.1 break Statement}\label{break-statement}

Exits the loop immediately.

\begin{Shaded}
\begin{Highlighting}[]
\ControlFlowTok{for}\NormalTok{ (i }\ControlFlowTok{in} \DecValTok{1}\SpecialCharTok{:}\DecValTok{10}\NormalTok{) \{}
  \ControlFlowTok{if}\NormalTok{ (i }\SpecialCharTok{==} \DecValTok{6}\NormalTok{) \{}
    \FunctionTok{print}\NormalTok{(}\StringTok{"Breaking at 6"}\NormalTok{)}
    \ControlFlowTok{break}
\NormalTok{  \}}
  \FunctionTok{print}\NormalTok{(i)}
\NormalTok{\}}
\end{Highlighting}
\end{Shaded}

\begin{verbatim}
#> [1] 1
#> [1] 2
#> [1] 3
#> [1] 4
#> [1] 5
#> [1] "Breaking at 6"
\end{verbatim}

\subsubsection{3.2 next Statement}\label{next-statement}

Skips the current iteration and continues with the next.

\begin{Shaded}
\begin{Highlighting}[]
\ControlFlowTok{for}\NormalTok{ (i }\ControlFlowTok{in} \DecValTok{1}\SpecialCharTok{:}\DecValTok{10}\NormalTok{) \{}
  \ControlFlowTok{if}\NormalTok{ (i }\SpecialCharTok{\%\%} \DecValTok{2} \SpecialCharTok{==} \DecValTok{0}\NormalTok{) \{}
    \ControlFlowTok{next}  \CommentTok{\# Skip even numbers}
\NormalTok{  \}}
  \FunctionTok{print}\NormalTok{(}\FunctionTok{paste}\NormalTok{(}\StringTok{"Odd number:"}\NormalTok{, i))}
\NormalTok{\}}
\end{Highlighting}
\end{Shaded}

\begin{verbatim}
#> [1] "Odd number: 1"
#> [1] "Odd number: 3"
#> [1] "Odd number: 5"
#> [1] "Odd number: 7"
#> [1] "Odd number: 9"
\end{verbatim}

\begin{center}\rule{0.5\linewidth}{0.5pt}\end{center}

\section{Part 2: Version Control
Systems}\label{part-2-version-control-systems}

\subsection{What is Version Control?}\label{what-is-version-control}

A \textbf{Version Control System (VCS)} is a tool that tracks and
manages changes to source code over time. It enables:

\begin{itemize}
\tightlist
\item
  \textbf{Change Tracking}: Record every modification to files
\item
  \textbf{Collaboration}: Multiple developers working simultaneously
\item
  \textbf{History Management}: Access to complete project evolution
\item
  \textbf{Reversion}: Ability to restore previous versions
\item
  \textbf{Branching}: Parallel development of features
\end{itemize}

\begin{center}\rule{0.5\linewidth}{0.5pt}\end{center}

\subsection{Components of Version Control
Systems}\label{components-of-version-control-systems}

\subsubsection{Key Concepts}\label{key-concepts}

\begin{enumerate}
\def\labelenumi{\arabic{enumi}.}
\tightlist
\item
  \textbf{Repository}: Central storage for all project files and their
  complete history
\item
  \textbf{Revision}: A specific saved version of files (identified by
  unique ID/hash)
\item
  \textbf{Branch}: Separate line of development for features or fixes
\item
  \textbf{Merging}: Combining changes from different branches
\item
  \textbf{Commit}: Snapshot of changes at a specific point in time
\end{enumerate}

\begin{center}\rule{0.5\linewidth}{0.5pt}\end{center}

\subsection{Types of Version Control
Systems}\label{types-of-version-control-systems}

\subsubsection{1. Local Version Control
Systems}\label{local-version-control-systems}

\begin{itemize}
\tightlist
\item
  Operates entirely on local machine
\item
  All history stored in local database
\item
  \textbf{Pros}: Simple, no network needed
\item
  \textbf{Cons}: No collaboration, single point of failure
\end{itemize}

\subsubsection{2. Centralized Version Control Systems
(CVCS)}\label{centralized-version-control-systems-cvcs}

Examples: \textbf{SVN (Subversion)}, CVS

\textbf{Architecture:} - Single central server stores all files -
Developers check out files from central location - Changes committed
directly to central server

\textbf{Workflow:} 1. \textbf{Checkout/Update}: Get latest files from
server 2. \textbf{Make Changes}: Edit files locally 3. \textbf{Commit}:
Save changes to central server

\textbf{Pros:} - Easy to understand and manage - Fine-grained access
control - Everyone sees what others are doing

\textbf{Cons:} - Single point of failure (server down = no work) -
Requires network connection - Limited offline capabilities

\subsubsection{3. Distributed Version Control Systems
(DVCS)}\label{distributed-version-control-systems-dvcs}

Examples: \textbf{Git}, Mercurial, Bazaar

\textbf{Architecture:} - Every developer has full repository copy -
Complete history available locally - Changes shared through push/pull

\textbf{Workflow:} 1. \textbf{Commit}: Save changes to local repository
2. \textbf{Push}: Upload changes to remote repository 3.
\textbf{Pull/Fetch}: Download changes from remote repository

\textbf{Pros:} - Fast operations (local) - Work offline - Multiple
backup copies - Flexible workflows - No single point of failure

\textbf{Cons:} - Steeper learning curve - More complex commands

\begin{center}\rule{0.5\linewidth}{0.5pt}\end{center}

\subsection{Popular Version Control
Systems}\label{popular-version-control-systems}

\subsubsection{1. Git ⭐ (Most Popular)}\label{git-most-popular}

\begin{itemize}
\tightlist
\item
  Created by Linus Torvalds (2005)
\item
  Distributed VCS
\item
  Fast, efficient, lightweight
\item
  Excellent branching and merging
\item
  Powers GitHub, GitLab, Bitbucket
\end{itemize}

\subsubsection{2. Subversion (SVN)}\label{subversion-svn}

\begin{itemize}
\tightlist
\item
  Centralized VCS
\item
  Still widely used in enterprises
\item
  Simpler than Git for basic use
\item
  Good for binary files
\end{itemize}

\subsubsection{3. Mercurial}\label{mercurial}

\begin{itemize}
\tightlist
\item
  Distributed VCS like Git
\item
  Simpler interface than Git
\item
  Good for large projects
\item
  Used by Facebook, Mozilla (historically)
\end{itemize}

\subsubsection{4. CVS (Concurrent Versions
System)}\label{cvs-concurrent-versions-system}

\begin{itemize}
\tightlist
\item
  One of the earliest VCS
\item
  Centralized system
\item
  Legacy system (mostly replaced)
\item
  Historical importance
\end{itemize}

\subsubsection{5. Bazaar}\label{bazaar}

\begin{itemize}
\tightlist
\item
  Developed by Canonical (Ubuntu creators)
\item
  Supports both centralized and distributed workflows
\item
  Beginner-friendly
\item
  No longer actively developed
\end{itemize}

\begin{center}\rule{0.5\linewidth}{0.5pt}\end{center}

\section{Part 3: Git - The Complete
Guide}\label{part-3-git---the-complete-guide}

\subsection{1. Introduction to Git}\label{introduction-to-git}

\textbf{Git} is an open-source \textbf{distributed version control
system} designed for:

\begin{itemize}
\tightlist
\item
  Tracking source code changes
\item
  Coordinating work among multiple developers
\item
  Managing project versions
\item
  Supporting non-linear development (branching)
\end{itemize}

\textbf{Created by}: Linus Torvalds (2005)\\
\textbf{Purpose}: Managing Linux kernel development\\
\textbf{License}: GPL v2 (Free and Open Source)

\begin{center}\rule{0.5\linewidth}{0.5pt}\end{center}

\subsection{2. Core Git Concepts}\label{core-git-concepts}

\subsubsection{Three States of Files}\label{three-states-of-files}

\begin{enumerate}
\def\labelenumi{\arabic{enumi}.}
\tightlist
\item
  \textbf{Working Directory}: Where you edit files
\item
  \textbf{Staging Area (Index)}: Files marked for next commit
\item
  \textbf{Repository (.git directory)}: Committed and saved permanently
\end{enumerate}

\begin{verbatim}
Working Directory → (git add) → Staging Area → (git commit) → Repository
\end{verbatim}

\subsubsection{Repository Types}\label{repository-types}

\begin{itemize}
\tightlist
\item
  \textbf{Local Repository}: On your machine
\item
  \textbf{Remote Repository}: Hosted server (GitHub, GitLab)
\item
  \textbf{Bare Repository}: No working directory (central repos)
\end{itemize}

\begin{center}\rule{0.5\linewidth}{0.5pt}\end{center}

\subsection{3. Git Environment Setup}\label{git-environment-setup}

\subsubsection{Installation}\label{installation}

\textbf{Windows}: Download Git for Windows (includes Git Bash)\\
\textbf{Mac}: \texttt{brew\ install\ git} or download from git-scm.com\\
\textbf{Linux}: \texttt{sudo\ apt-get\ install\ git} (Ubuntu/Debian)

\subsubsection{Initial Configuration}\label{initial-configuration}

\begin{Shaded}
\begin{Highlighting}[]
\CommentTok{\# Set username}
\FunctionTok{git}\NormalTok{ config }\AttributeTok{{-}{-}global}\NormalTok{ user.name }\StringTok{"Your Name"}

\CommentTok{\# Set email}
\FunctionTok{git}\NormalTok{ config }\AttributeTok{{-}{-}global}\NormalTok{ user.email }\StringTok{"your.email@example.com"}

\CommentTok{\# Check configuration}
\FunctionTok{git}\NormalTok{ config }\AttributeTok{{-}{-}list}

\CommentTok{\# Set default editor}
\FunctionTok{git}\NormalTok{ config }\AttributeTok{{-}{-}global}\NormalTok{ core.editor }\StringTok{"vim"}

\CommentTok{\# Set default branch name}
\FunctionTok{git}\NormalTok{ config }\AttributeTok{{-}{-}global}\NormalTok{ init.defaultBranch main}
\end{Highlighting}
\end{Shaded}

\begin{center}\rule{0.5\linewidth}{0.5pt}\end{center}

\subsection{4. Basic Git Commands}\label{basic-git-commands}

\subsubsection{Initializing a
Repository}\label{initializing-a-repository}

\begin{Shaded}
\begin{Highlighting}[]
\CommentTok{\# Create new repository}
\FunctionTok{git}\NormalTok{ init}

\CommentTok{\# Clone existing repository}
\FunctionTok{git}\NormalTok{ clone }\OperatorTok{\textless{}}\NormalTok{repository{-}url}\OperatorTok{\textgreater{}}
\FunctionTok{git}\NormalTok{ clone https://github.com/username/repo.git}
\end{Highlighting}
\end{Shaded}

\subsubsection{Checking Status}\label{checking-status}

\begin{Shaded}
\begin{Highlighting}[]
\CommentTok{\# View status of files}
\FunctionTok{git}\NormalTok{ status}

\CommentTok{\# Short status}
\FunctionTok{git}\NormalTok{ status }\AttributeTok{{-}s}
\end{Highlighting}
\end{Shaded}

\subsubsection{Adding Files to Staging}\label{adding-files-to-staging}

\begin{Shaded}
\begin{Highlighting}[]
\CommentTok{\# Add specific file}
\FunctionTok{git}\NormalTok{ add filename.R}

\CommentTok{\# Add all files}
\FunctionTok{git}\NormalTok{ add .}

\CommentTok{\# Add all R files}
\FunctionTok{git}\NormalTok{ add }\PreprocessorTok{*}\NormalTok{.R}

\CommentTok{\# Add files interactively}
\FunctionTok{git}\NormalTok{ add }\AttributeTok{{-}i}
\end{Highlighting}
\end{Shaded}

\subsubsection{Committing Changes}\label{committing-changes}

\begin{Shaded}
\begin{Highlighting}[]
\CommentTok{\# Commit with message}
\FunctionTok{git}\NormalTok{ commit }\AttributeTok{{-}m} \StringTok{"Your commit message"}

\CommentTok{\# Commit all tracked files (skip staging)}
\FunctionTok{git}\NormalTok{ commit }\AttributeTok{{-}a} \AttributeTok{{-}m} \StringTok{"Commit message"}

\CommentTok{\# Amend last commit}
\FunctionTok{git}\NormalTok{ commit }\AttributeTok{{-}{-}amend} \AttributeTok{{-}m} \StringTok{"Updated message"}
\end{Highlighting}
\end{Shaded}

\subsubsection{Viewing History}\label{viewing-history}

\begin{Shaded}
\begin{Highlighting}[]
\CommentTok{\# View commit history}
\FunctionTok{git}\NormalTok{ log}

\CommentTok{\# Compact view}
\FunctionTok{git}\NormalTok{ log }\AttributeTok{{-}{-}oneline}

\CommentTok{\# View last 5 commits}
\FunctionTok{git}\NormalTok{ log }\AttributeTok{{-}5}

\CommentTok{\# View with graph}
\FunctionTok{git}\NormalTok{ log }\AttributeTok{{-}{-}graph} \AttributeTok{{-}{-}oneline} \AttributeTok{{-}{-}all}

\CommentTok{\# View specific file history}
\FunctionTok{git}\NormalTok{ log }\AttributeTok{{-}{-}}\NormalTok{ filename.R}
\end{Highlighting}
\end{Shaded}

\begin{center}\rule{0.5\linewidth}{0.5pt}\end{center}

\subsection{5. Working with Remote
Repositories}\label{working-with-remote-repositories}

\subsubsection{Managing Remotes}\label{managing-remotes}

\begin{Shaded}
\begin{Highlighting}[]
\CommentTok{\# View remote repositories}
\FunctionTok{git}\NormalTok{ remote }\AttributeTok{{-}v}

\CommentTok{\# Add remote repository}
\FunctionTok{git}\NormalTok{ remote add origin https://github.com/username/repo.git}

\CommentTok{\# Remove remote}
\FunctionTok{git}\NormalTok{ remote remove origin}

\CommentTok{\# Rename remote}
\FunctionTok{git}\NormalTok{ remote rename origin upstream}
\end{Highlighting}
\end{Shaded}

\subsubsection{Pushing Changes}\label{pushing-changes}

\begin{Shaded}
\begin{Highlighting}[]
\CommentTok{\# Push to remote}
\FunctionTok{git}\NormalTok{ push origin main}

\CommentTok{\# Push and set upstream}
\FunctionTok{git}\NormalTok{ push }\AttributeTok{{-}u}\NormalTok{ origin main}

\CommentTok{\# Push all branches}
\FunctionTok{git}\NormalTok{ push }\AttributeTok{{-}{-}all}

\CommentTok{\# Force push (use with caution!)}
\FunctionTok{git}\NormalTok{ push }\AttributeTok{{-}{-}force}
\end{Highlighting}
\end{Shaded}

\subsubsection{Pulling Changes}\label{pulling-changes}

\begin{Shaded}
\begin{Highlighting}[]
\CommentTok{\# Fetch and merge changes}
\FunctionTok{git}\NormalTok{ pull origin main}

\CommentTok{\# Fetch without merging}
\FunctionTok{git}\NormalTok{ fetch origin}

\CommentTok{\# Pull with rebase}
\FunctionTok{git}\NormalTok{ pull }\AttributeTok{{-}{-}rebase}\NormalTok{ origin main}
\end{Highlighting}
\end{Shaded}

\begin{center}\rule{0.5\linewidth}{0.5pt}\end{center}

\subsection{6. Branching and Merging}\label{branching-and-merging}

\subsubsection{Branch Management}\label{branch-management}

\begin{Shaded}
\begin{Highlighting}[]
\CommentTok{\# Create new branch}
\FunctionTok{git}\NormalTok{ branch feature{-}branch}

\CommentTok{\# Switch to branch}
\FunctionTok{git}\NormalTok{ checkout feature{-}branch}

\CommentTok{\# Create and switch (shortcut)}
\FunctionTok{git}\NormalTok{ checkout }\AttributeTok{{-}b}\NormalTok{ feature{-}branch}

\CommentTok{\# Modern way (Git 2.23+)}
\FunctionTok{git}\NormalTok{ switch feature{-}branch}
\FunctionTok{git}\NormalTok{ switch }\AttributeTok{{-}c}\NormalTok{ new{-}branch}

\CommentTok{\# List branches}
\FunctionTok{git}\NormalTok{ branch}
\FunctionTok{git}\NormalTok{ branch }\AttributeTok{{-}a}  \CommentTok{\# include remote branches}

\CommentTok{\# Delete branch}
\FunctionTok{git}\NormalTok{ branch }\AttributeTok{{-}d}\NormalTok{ feature{-}branch}
\FunctionTok{git}\NormalTok{ branch }\AttributeTok{{-}D}\NormalTok{ feature{-}branch  }\CommentTok{\# force delete}

\CommentTok{\# Delete remote branch}
\FunctionTok{git}\NormalTok{ push origin }\AttributeTok{{-}{-}delete}\NormalTok{ feature{-}branch}
\end{Highlighting}
\end{Shaded}

\subsubsection{Merging}\label{merging}

\begin{Shaded}
\begin{Highlighting}[]
\CommentTok{\# Merge branch into current branch}
\FunctionTok{git}\NormalTok{ merge feature{-}branch}

\CommentTok{\# Merge without fast{-}forward}
\FunctionTok{git}\NormalTok{ merge }\AttributeTok{{-}{-}no{-}ff}\NormalTok{ feature{-}branch}

\CommentTok{\# Abort merge}
\FunctionTok{git}\NormalTok{ merge }\AttributeTok{{-}{-}abort}
\end{Highlighting}
\end{Shaded}

\subsubsection{Handling Merge Conflicts}\label{handling-merge-conflicts}

\begin{Shaded}
\begin{Highlighting}[]
\CommentTok{\# 1. Git will mark conflicts in files}
\CommentTok{\# 2. Open conflicted files and resolve manually}
\CommentTok{\# 3. Add resolved files}
\FunctionTok{git}\NormalTok{ add resolved{-}file.R}

\CommentTok{\# 4. Complete merge}
\FunctionTok{git}\NormalTok{ commit}
\end{Highlighting}
\end{Shaded}

\subsubsection{Rebasing}\label{rebasing}

\begin{Shaded}
\begin{Highlighting}[]
\CommentTok{\# Rebase current branch onto main}
\FunctionTok{git}\NormalTok{ rebase main}

\CommentTok{\# Interactive rebase (last 3 commits)}
\FunctionTok{git}\NormalTok{ rebase }\AttributeTok{{-}i}\NormalTok{ HEAD\textasciitilde{}3}

\CommentTok{\# Continue after resolving conflicts}
\FunctionTok{git}\NormalTok{ rebase }\AttributeTok{{-}{-}continue}

\CommentTok{\# Abort rebase}
\FunctionTok{git}\NormalTok{ rebase }\AttributeTok{{-}{-}abort}
\end{Highlighting}
\end{Shaded}

\begin{center}\rule{0.5\linewidth}{0.5pt}\end{center}

\subsection{7. Advanced Git Commands}\label{advanced-git-commands}

\subsubsection{Viewing Differences}\label{viewing-differences}

\begin{Shaded}
\begin{Highlighting}[]
\CommentTok{\# Changes in working directory}
\FunctionTok{git}\NormalTok{ diff}

\CommentTok{\# Changes in staging area}
\FunctionTok{git}\NormalTok{ diff }\AttributeTok{{-}{-}staged}

\CommentTok{\# Difference between branches}
\FunctionTok{git}\NormalTok{ diff main feature{-}branch}

\CommentTok{\# Difference for specific file}
\FunctionTok{git}\NormalTok{ diff filename.R}
\end{Highlighting}
\end{Shaded}

\subsubsection{Undoing Changes}\label{undoing-changes}

\begin{Shaded}
\begin{Highlighting}[]
\CommentTok{\# Discard changes in working directory}
\FunctionTok{git}\NormalTok{ restore filename.R}
\FunctionTok{git}\NormalTok{ checkout }\AttributeTok{{-}{-}}\NormalTok{ filename.R  }\CommentTok{\# old way}

\CommentTok{\# Unstage file}
\FunctionTok{git}\NormalTok{ restore }\AttributeTok{{-}{-}staged}\NormalTok{ filename.R}
\FunctionTok{git}\NormalTok{ reset HEAD filename.R  }\CommentTok{\# old way}

\CommentTok{\# Revert a commit}
\FunctionTok{git}\NormalTok{ revert }\OperatorTok{\textless{}}\NormalTok{commit{-}hash}\OperatorTok{\textgreater{}}

\CommentTok{\# Reset to previous commit}
\FunctionTok{git}\NormalTok{ reset }\AttributeTok{{-}{-}soft}\NormalTok{ HEAD\textasciitilde{}1  }\CommentTok{\# keep changes}
\FunctionTok{git}\NormalTok{ reset }\AttributeTok{{-}{-}mixed}\NormalTok{ HEAD\textasciitilde{}1  }\CommentTok{\# unstage changes}
\FunctionTok{git}\NormalTok{ reset }\AttributeTok{{-}{-}hard}\NormalTok{ HEAD\textasciitilde{}1  }\CommentTok{\# discard changes (⚠️ dangerous!)}
\end{Highlighting}
\end{Shaded}

\subsubsection{Stashing Changes}\label{stashing-changes}

\begin{Shaded}
\begin{Highlighting}[]
\CommentTok{\# Save work temporarily}
\FunctionTok{git}\NormalTok{ stash}

\CommentTok{\# List stashes}
\FunctionTok{git}\NormalTok{ stash list}

\CommentTok{\# Apply latest stash}
\FunctionTok{git}\NormalTok{ stash apply}

\CommentTok{\# Apply and remove stash}
\FunctionTok{git}\NormalTok{ stash pop}

\CommentTok{\# Stash with message}
\FunctionTok{git}\NormalTok{ stash save }\StringTok{"Work in progress"}

\CommentTok{\# Clear all stashes}
\FunctionTok{git}\NormalTok{ stash clear}
\end{Highlighting}
\end{Shaded}

\subsubsection{Tagging}\label{tagging}

\begin{Shaded}
\begin{Highlighting}[]
\CommentTok{\# Create lightweight tag}
\FunctionTok{git}\NormalTok{ tag v1.0}

\CommentTok{\# Create annotated tag}
\FunctionTok{git}\NormalTok{ tag }\AttributeTok{{-}a}\NormalTok{ v1.0 }\AttributeTok{{-}m} \StringTok{"Version 1.0 release"}

\CommentTok{\# List tags}
\FunctionTok{git}\NormalTok{ tag}

\CommentTok{\# Push tag to remote}
\FunctionTok{git}\NormalTok{ push origin v1.0}

\CommentTok{\# Push all tags}
\FunctionTok{git}\NormalTok{ push }\AttributeTok{{-}{-}tags}

\CommentTok{\# Delete tag}
\FunctionTok{git}\NormalTok{ tag }\AttributeTok{{-}d}\NormalTok{ v1.0}
\FunctionTok{git}\NormalTok{ push origin }\AttributeTok{{-}{-}delete}\NormalTok{ v1.0}
\end{Highlighting}
\end{Shaded}

\begin{center}\rule{0.5\linewidth}{0.5pt}\end{center}

\subsection{8. Git Best Practices}\label{git-best-practices}

\subsubsection{Commit Messages}\label{commit-messages}

\textbf{Good commit message format:}

\begin{verbatim}
Short summary (50 chars or less)

Detailed explanation if needed. Wrap at 72 characters.
Explain what and why, not how.

- Bullet points are okay
- Use present tense: "Add feature" not "Added feature"
\end{verbatim}

\textbf{Examples:}

\begin{Shaded}
\begin{Highlighting}[]
\FunctionTok{git}\NormalTok{ commit }\AttributeTok{{-}m} \StringTok{"Add data validation function for user input"}
\FunctionTok{git}\NormalTok{ commit }\AttributeTok{{-}m} \StringTok{"Fix bug in date parsing for CSV imports"}
\FunctionTok{git}\NormalTok{ commit }\AttributeTok{{-}m} \StringTok{"Update README with installation instructions"}
\end{Highlighting}
\end{Shaded}

\subsubsection{Branching Strategy}\label{branching-strategy}

\textbf{Common strategies:}

\begin{enumerate}
\def\labelenumi{\arabic{enumi}.}
\tightlist
\item
  \textbf{Git Flow}:

  \begin{itemize}
  \tightlist
  \item
    \texttt{main}: Production code
  \item
    \texttt{develop}: Integration branch
  \item
    \texttt{feature/*}: New features
  \item
    \texttt{hotfix/*}: Emergency fixes
  \end{itemize}
\item
  \textbf{GitHub Flow}:

  \begin{itemize}
  \tightlist
  \item
    \texttt{main}: Always deployable
  \item
    \texttt{feature/*}: All development
  \end{itemize}
\item
  \textbf{Trunk-Based Development}:

  \begin{itemize}
  \tightlist
  \item
    \texttt{main}: Single integration branch
  \item
    Short-lived feature branches
  \end{itemize}
\end{enumerate}

\begin{center}\rule{0.5\linewidth}{0.5pt}\end{center}

\subsection{9. Git with R Programming}\label{git-with-r-programming}

\subsubsection{Tracking R Projects}\label{tracking-r-projects}

\begin{Shaded}
\begin{Highlighting}[]
\CommentTok{\# Initialize R project with Git}
\FunctionTok{git}\NormalTok{ init}

\CommentTok{\# Create .gitignore for R}
\BuiltInTok{echo} \StringTok{".Rproj.user}
\StringTok{.Rhistory}
\StringTok{.RData}
\StringTok{.Ruserdata}
\StringTok{*.Rproj}
\StringTok{.DS\_Store"} \OperatorTok{\textgreater{}}\NormalTok{ .gitignore}

\CommentTok{\# Add files}
\FunctionTok{git}\NormalTok{ add .}
\FunctionTok{git}\NormalTok{ commit }\AttributeTok{{-}m} \StringTok{"Initial commit of R project"}
\end{Highlighting}
\end{Shaded}

\subsubsection{Example .gitignore for R
Projects}\label{example-.gitignore-for-r-projects}

\begin{verbatim}
# R specific
.Rproj.user
.Rhistory
.RData
.Ruserdata
*.Rproj

# Data files (if large)
*.csv
*.xlsx
data/

# Output files
*.pdf
*.png
figures/

# OS specific
.DS_Store
Thumbs.db
\end{verbatim}

\subsubsection{Using Git with RStudio}\label{using-git-with-rstudio}

RStudio has built-in Git integration:

\begin{enumerate}
\def\labelenumi{\arabic{enumi}.}
\tightlist
\item
  \textbf{Tools} → \textbf{Version Control} → \textbf{Project Setup}
\item
  Use Git pane for staging, committing
\item
  Visual diff viewer
\item
  Branch management
\item
  Push/pull buttons
\end{enumerate}

\begin{center}\rule{0.5\linewidth}{0.5pt}\end{center}

\section{Part 4: GitHub}\label{part-4-github}

\subsection{What is GitHub?}\label{what-is-github}

\textbf{GitHub} is a web-based platform for hosting Git repositories and
collaborating on code.

\subsubsection{Git vs GitHub}\label{git-vs-github}

\begin{longtable}[]{@{}lll@{}}
\toprule\noalign{}
Feature & Git & GitHub \\
\midrule\noalign{}
\endhead
\bottomrule\noalign{}
\endlastfoot
Type & Version control tool & Hosting service \\
Location & Local computer & Cloud-based \\
Purpose & Track changes & Collaboration platform \\
Access & Command line/GUI & Web interface + Git \\
Created by & Linus Torvalds & Chris Wanstrath, et al. \\
Owned by & Linux community & Microsoft \\
\end{longtable}

\begin{center}\rule{0.5\linewidth}{0.5pt}\end{center}

\subsection{GitHub Features}\label{github-features}

\subsubsection{Core Features}\label{core-features}

\begin{enumerate}
\def\labelenumi{\arabic{enumi}.}
\tightlist
\item
  \textbf{Repository Hosting}: Store and manage Git repositories
\item
  \textbf{Collaboration}: Pull requests, code reviews
\item
  \textbf{Issue Tracking}: Bug reports, feature requests
\item
  \textbf{Project Management}: Kanban boards, milestones
\item
  \textbf{Actions}: CI/CD automation
\item
  \textbf{Pages}: Host static websites
\item
  \textbf{Wikis}: Documentation
\item
  \textbf{Security}: Dependency scanning, secret detection
\end{enumerate}

\begin{center}\rule{0.5\linewidth}{0.5pt}\end{center}

\subsection{Working with GitHub}\label{working-with-github}

\subsubsection{Creating a Repository}\label{creating-a-repository}

\textbf{On GitHub:} 1. Click ``New repository'' 2. Enter repository name
3. Add description (optional) 4. Choose public/private 5. Initialize
with README (optional) 6. Add .gitignore template 7. Choose license 8.
Click ``Create repository''

\textbf{From Command Line:}

\begin{Shaded}
\begin{Highlighting}[]
\CommentTok{\# Create local repo}
\FunctionTok{git}\NormalTok{ init}
\FunctionTok{git}\NormalTok{ add .}
\FunctionTok{git}\NormalTok{ commit }\AttributeTok{{-}m} \StringTok{"Initial commit"}

\CommentTok{\# Connect to GitHub}
\FunctionTok{git}\NormalTok{ remote add origin https://github.com/username/repo.git}
\FunctionTok{git}\NormalTok{ push }\AttributeTok{{-}u}\NormalTok{ origin main}
\end{Highlighting}
\end{Shaded}

\subsubsection{Cloning a Repository}\label{cloning-a-repository}

\begin{Shaded}
\begin{Highlighting}[]
\CommentTok{\# Clone repository}
\FunctionTok{git}\NormalTok{ clone https://github.com/username/repo.git}

\CommentTok{\# Clone specific branch}
\FunctionTok{git}\NormalTok{ clone }\AttributeTok{{-}b}\NormalTok{ branch{-}name https://github.com/username/repo.git}

\CommentTok{\# Clone to specific folder}
\FunctionTok{git}\NormalTok{ clone https://github.com/username/repo.git my{-}folder}
\end{Highlighting}
\end{Shaded}

\subsubsection{Forking Workflow}\label{forking-workflow}

\textbf{Fork} = Personal copy of someone else's repository

\begin{enumerate}
\def\labelenumi{\arabic{enumi}.}
\tightlist
\item
  Click ``Fork'' on GitHub
\item
  Clone your fork
\end{enumerate}

\begin{Shaded}
\begin{Highlighting}[]
\FunctionTok{git}\NormalTok{ clone https://github.com/your{-}username/repo.git}
\BuiltInTok{cd}\NormalTok{ repo}
\end{Highlighting}
\end{Shaded}

\begin{enumerate}
\def\labelenumi{\arabic{enumi}.}
\setcounter{enumi}{2}
\tightlist
\item
  Add upstream remote
\end{enumerate}

\begin{Shaded}
\begin{Highlighting}[]
\FunctionTok{git}\NormalTok{ remote add upstream https://github.com/original{-}owner/repo.git}
\end{Highlighting}
\end{Shaded}

\begin{enumerate}
\def\labelenumi{\arabic{enumi}.}
\setcounter{enumi}{3}
\tightlist
\item
  Keep fork updated
\end{enumerate}

\begin{Shaded}
\begin{Highlighting}[]
\FunctionTok{git}\NormalTok{ fetch upstream}
\FunctionTok{git}\NormalTok{ merge upstream/main}
\end{Highlighting}
\end{Shaded}

\subsubsection{Pull Requests}\label{pull-requests}

\textbf{Creating a Pull Request:}

\begin{enumerate}
\def\labelenumi{\arabic{enumi}.}
\tightlist
\item
  Create and push feature branch
\end{enumerate}

\begin{Shaded}
\begin{Highlighting}[]
\FunctionTok{git}\NormalTok{ checkout }\AttributeTok{{-}b}\NormalTok{ feature{-}branch}
\CommentTok{\# make changes}
\FunctionTok{git}\NormalTok{ push origin feature{-}branch}
\end{Highlighting}
\end{Shaded}

\begin{enumerate}
\def\labelenumi{\arabic{enumi}.}
\setcounter{enumi}{1}
\tightlist
\item
  Go to GitHub repository
\item
  Click ``New pull request''
\item
  Select base and compare branches
\item
  Add title and description
\item
  Click ``Create pull request''
\end{enumerate}

\textbf{Reviewing Pull Requests:} - Review code changes - Add comments -
Approve or request changes - Merge when approved

\begin{center}\rule{0.5\linewidth}{0.5pt}\end{center}

\subsection{GitHub Best Practices}\label{github-best-practices}

\subsubsection{README.md Template}\label{readme.md-template}

\begin{Shaded}
\begin{Highlighting}[]
\FunctionTok{\# Project Title}

\NormalTok{Brief description of project}

\FunctionTok{\#\# Installation}

\SpecialCharTok{\textbackslash{}\textasciigrave{}\textbackslash{}\textasciigrave{}\textbackslash{}\textasciigrave{}}\NormalTok{r}
\NormalTok{install.packages("packagename")}
\SpecialCharTok{\textbackslash{}\textasciigrave{}\textbackslash{}\textasciigrave{}\textbackslash{}\textasciigrave{}}

\FunctionTok{\#\# Usage}

\SpecialCharTok{\textbackslash{}\textasciigrave{}\textbackslash{}\textasciigrave{}\textbackslash{}\textasciigrave{}}\NormalTok{r}
\NormalTok{library(packagename)}
\FunctionTok{\# example code}
\SpecialCharTok{\textbackslash{}\textasciigrave{}\textbackslash{}\textasciigrave{}\textbackslash{}\textasciigrave{}}

\FunctionTok{\#\# Contributing}

\NormalTok{Pull requests welcome!}

\FunctionTok{\#\# License}

\NormalTok{MIT License}
\end{Highlighting}
\end{Shaded}

\subsubsection{Contributing Guidelines}\label{contributing-guidelines}

\begin{enumerate}
\def\labelenumi{\arabic{enumi}.}
\tightlist
\item
  \textbf{Fork} the repository
\item
  \textbf{Create} feature branch
  (\texttt{git\ checkout\ -b\ feature/AmazingFeature})
\item
  \textbf{Commit} changes
  (\texttt{git\ commit\ -m\ \textquotesingle{}Add\ some\ AmazingFeature\textquotesingle{}})
\item
  \textbf{Push} to branch
  (\texttt{git\ push\ origin\ feature/AmazingFeature})
\item
  \textbf{Open} Pull Request
\end{enumerate}

\begin{center}\rule{0.5\linewidth}{0.5pt}\end{center}

\section{Part 5: Git in CI/CD and
Deployment}\label{part-5-git-in-cicd-and-deployment}

\subsection{Continuous Integration/Continuous
Deployment}\label{continuous-integrationcontinuous-deployment}

\subsubsection{What is CI/CD?}\label{what-is-cicd}

\begin{itemize}
\tightlist
\item
  \textbf{CI}: Automatically test code when pushed
\item
  \textbf{CD}: Automatically deploy code when tests pass
\end{itemize}

\subsubsection{GitHub Actions Example}\label{github-actions-example}

\begin{Shaded}
\begin{Highlighting}[]
\CommentTok{\# .github/workflows/r{-}check.yml}
\FunctionTok{name}\KeywordTok{:}\AttributeTok{ R{-}CMD{-}check}

\FunctionTok{on}\KeywordTok{:}\AttributeTok{ }\KeywordTok{[}\AttributeTok{push}\KeywordTok{,}\AttributeTok{ pull\_request}\KeywordTok{]}

\FunctionTok{jobs}\KeywordTok{:}
\AttributeTok{  }\FunctionTok{R{-}CMD{-}check}\KeywordTok{:}
\AttributeTok{    }\FunctionTok{runs{-}on}\KeywordTok{:}\AttributeTok{ ubuntu{-}latest}
\AttributeTok{    }\FunctionTok{steps}\KeywordTok{:}
\AttributeTok{      }\KeywordTok{{-}}\AttributeTok{ }\FunctionTok{uses}\KeywordTok{:}\AttributeTok{ actions/checkout@v2}
\AttributeTok{      }\KeywordTok{{-}}\AttributeTok{ }\FunctionTok{uses}\KeywordTok{:}\AttributeTok{ r{-}lib/actions/setup{-}r@v2}
\AttributeTok{      }\KeywordTok{{-}}\AttributeTok{ }\FunctionTok{name}\KeywordTok{:}\AttributeTok{ Install dependencies}
\FunctionTok{        run}\KeywordTok{: }\CharTok{|}
\NormalTok{          install.packages(c("remotes", "rcmdcheck"))}
\AttributeTok{        }\FunctionTok{shell}\KeywordTok{:}\AttributeTok{ Rscript \{0\}}
\AttributeTok{      }\KeywordTok{{-}}\AttributeTok{ }\FunctionTok{name}\KeywordTok{:}\AttributeTok{ Check}
\AttributeTok{        }\FunctionTok{run}\KeywordTok{:}\AttributeTok{ rcmdcheck::rcmdcheck(args = "{-}{-}no{-}manual", error\_on = "error")}
\AttributeTok{        }\FunctionTok{shell}\KeywordTok{:}\AttributeTok{ Rscript \{0\}}
\end{Highlighting}
\end{Shaded}

\begin{center}\rule{0.5\linewidth}{0.5pt}\end{center}

\section{Summary and Key Takeaways}\label{summary-and-key-takeaways}

\subsection{Control Flow}\label{control-flow}

\begin{itemize}
\tightlist
\item
  \textbf{if/else}: Decision making based on conditions
\item
  \textbf{for loop}: Known number of iterations
\item
  \textbf{while loop}: Condition-based repetition
\item
  \textbf{repeat loop}: Infinite loop with manual break
\item
  \textbf{switch}: Multiple case handling
\end{itemize}

\subsection{Version Control}\label{version-control}

\begin{itemize}
\tightlist
\item
  \textbf{Local VCS}: Single machine, no collaboration
\item
  \textbf{Centralized VCS}: Single server, central authority (SVN)
\item
  \textbf{Distributed VCS}: Multiple copies, flexible workflow (Git)
\end{itemize}

\subsection{Git Essentials}\label{git-essentials}

\begin{Shaded}
\begin{Highlighting}[]
\CommentTok{\# Basic workflow}
\FunctionTok{git}\NormalTok{ init                    }\CommentTok{\# Initialize}
\FunctionTok{git}\NormalTok{ add .                   }\CommentTok{\# Stage changes}
\FunctionTok{git}\NormalTok{ commit }\AttributeTok{{-}m} \StringTok{"message"}     \CommentTok{\# Commit}
\FunctionTok{git}\NormalTok{ push origin main        }\CommentTok{\# Push to remote}
\FunctionTok{git}\NormalTok{ pull origin main        }\CommentTok{\# Pull from remote}

\CommentTok{\# Branching}
\FunctionTok{git}\NormalTok{ checkout }\AttributeTok{{-}b}\NormalTok{ feature     }\CommentTok{\# Create branch}
\FunctionTok{git}\NormalTok{ merge feature           }\CommentTok{\# Merge branch}
\FunctionTok{git}\NormalTok{ branch }\AttributeTok{{-}d}\NormalTok{ feature       }\CommentTok{\# Delete branch}
\end{Highlighting}
\end{Shaded}

\subsection{GitHub Workflow}\label{github-workflow}

\begin{enumerate}
\def\labelenumi{\arabic{enumi}.}
\tightlist
\item
  \textbf{Fork} repository
\item
  \textbf{Clone} to local
\item
  \textbf{Branch} for features
\item
  \textbf{Commit} changes
\item
  \textbf{Push} to GitHub
\item
  \textbf{Pull Request} for review
\item
  \textbf{Merge} after approval
\end{enumerate}

\begin{center}\rule{0.5\linewidth}{0.5pt}\end{center}

\section{Practical Exercises}\label{practical-exercises}

\subsection{Exercise 1: Control Flow}\label{exercise-1-control-flow}

Create a grade calculator that: - Takes a score input - Uses if-else-if
to assign letter grade - Uses switch to assign GPA - Prints formatted
result

\subsection{Exercise 2: Loops}\label{exercise-2-loops}

Write R code to: - Calculate Fibonacci sequence (for loop) - Find prime
numbers (while loop) - Create menu system (repeat loop)

\subsection{Exercise 3: Git Practice}\label{exercise-3-git-practice}

\begin{enumerate}
\def\labelenumi{\arabic{enumi}.}
\tightlist
\item
  Create new R project
\item
  Initialize Git repository
\item
  Create multiple commits
\item
  Create and merge feature branch
\item
  Push to GitHub
\end{enumerate}

\subsection{Exercise 4: Collaboration}\label{exercise-4-collaboration}

\begin{enumerate}
\def\labelenumi{\arabic{enumi}.}
\tightlist
\item
  Fork a public R repository
\item
  Clone to local
\item
  Create feature branch
\item
  Make improvements
\item
  Submit pull request
\end{enumerate}

\begin{center}\rule{0.5\linewidth}{0.5pt}\end{center}

\section{Additional Resources}\label{additional-resources}

\subsection{Documentation}\label{documentation}

\begin{itemize}
\tightlist
\item
  \href{https://git-scm.com/doc}{Git Official Documentation}
\item
  \href{https://guides.github.com/}{GitHub Guides}
\item
  \href{https://r4ds.had.co.nz/}{R for Data Science}
\item
  \href{https://happygitwithr.com/}{Happy Git with R}
\end{itemize}

\subsection{Cheat Sheets}\label{cheat-sheets}

\begin{itemize}
\tightlist
\item
  \href{https://education.github.com/git-cheat-sheet-education.pdf}{Git
  Cheat Sheet}
\item
  \href{https://guides.github.com/introduction/flow/}{GitHub Flow}
\item
  \href{https://cran.r-project.org/doc/contrib/Short-refcard.pdf}{R
  Programming Cheat Sheet}
\end{itemize}

\subsection{Online Tutorials}\label{online-tutorials}

\begin{itemize}
\tightlist
\item
  \href{https://learngitbranching.js.org/}{Learn Git Branching}
\item
  \href{https://lab.github.com/}{GitHub Learning Lab}
\item
  \href{https://www.codecademy.com/learn/learn-git}{Codecademy Git
  Course}
\end{itemize}

\begin{center}\rule{0.5\linewidth}{0.5pt}\end{center}

\begin{Shaded}
\begin{Highlighting}[]
\CommentTok{\# Session Info}
\FunctionTok{sessionInfo}\NormalTok{()}
\end{Highlighting}
\end{Shaded}

\begin{verbatim}
#> R version 4.5.2 (2025-10-31 ucrt)
#> Platform: x86_64-w64-mingw32/x64
#> Running under: Windows 11 x64 (build 26200)
#> 
#> Matrix products: default
#>   LAPACK version 3.12.1
#> 
#> locale:
#> [1] LC_COLLATE=English_Kenya.utf8  LC_CTYPE=English_Kenya.utf8   
#> [3] LC_MONETARY=English_Kenya.utf8 LC_NUMERIC=C                  
#> [5] LC_TIME=English_Kenya.utf8    
#> 
#> time zone: Africa/Nairobi
#> tzcode source: internal
#> 
#> attached base packages:
#> [1] stats     graphics  grDevices utils     datasets  methods   base     
#> 
#> loaded via a namespace (and not attached):
#>  [1] compiler_4.5.2    fastmap_1.2.0     cli_3.6.5         tools_4.5.2      
#>  [5] htmltools_0.5.9   rstudioapi_0.17.1 yaml_2.3.12       rmarkdown_2.30   
#>  [9] knitr_1.51        xfun_0.55         digest_0.6.39     rlang_1.1.6      
#> [13] evaluate_1.0.5
\end{verbatim}

\begin{center}\rule{0.5\linewidth}{0.5pt}\end{center}

\textbf{End of Notes}

\emph{Created with R Markdown} 📝

\end{document}
